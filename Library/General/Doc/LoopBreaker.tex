\XtoCBlock{LoopBreaker}
\label{block:LoopBreaker}
\begin{figure}[H]\includegraphics{LoopBreaker}\end{figure} 

\begin{XtoCtabular}{Inports}
In & Input In(k)\tabularnewline
\hline
\end{XtoCtabular}


\begin{XtoCtabular}{Outports}
Out & Output Out(k)=In(k-1)\tabularnewline
\hline
\end{XtoCtabular}

\subsubsection*{Description:}
Block to break algebraic loops.

% include optional documentation file
\InputIfFileExists{\XcHomePath/Library/General/Doc/LoopBreaker_Info.tex}{\vspace{1ex}}{}

\subsubsection*{Implementations:}
\begin{tabular}{l l}
\textbf{FiP16} & 16 Bit Fixed Point Implementation\tabularnewline
\textbf{FiP32} & 32 Bit Fixed Point Implementation\tabularnewline
\textbf{Float32} & 32 Bit Floating Point Implementation\tabularnewline
\textbf{Float64} & 64 Bit Floating Point Implementation\tabularnewline
\end{tabular}

\XtoCImplementation{FiP16}
\index{Block ID!481}
\nopagebreak[0]
% Implementation details
\begin{tabular}{l l}
\textbf{Name} & FiP16 \tabularnewline
\textbf{ID} & 481 \tabularnewline
\textbf{Revision} & 0.1 \tabularnewline
\textbf{C filename} & LoopBreaker\_FiP16.c \tabularnewline
\textbf{H filename} & LoopBreaker\_FiP16.h \tabularnewline
\end{tabular}
\vspace{1ex}

16 Bit Fixed Point Implementation

% Implementation data structure
\XtoCDataStruct{Data Structure:}
\begin{lstlisting}
typedef struct {
     uint16        ID;
     int16         *In;
     int16         Out;
} LOOPBREAKER_FIP16;
\end{lstlisting}

\ifdefined \AddTestReports
\InputIfFileExists{\XcHomePath/Library/General/Doc/Test_LoopBreaker_FiP16.tex}{}{}
\fi
\XtoCImplementation{FiP32}
\index{Block ID!482}
\nopagebreak[0]
% Implementation details
\begin{tabular}{l l}
\textbf{Name} & FiP32 \tabularnewline
\textbf{ID} & 482 \tabularnewline
\textbf{Revision} & 0.1 \tabularnewline
\textbf{C filename} & LoopBreaker\_FiP32.c \tabularnewline
\textbf{H filename} & LoopBreaker\_FiP32.h \tabularnewline
\end{tabular}
\vspace{1ex}

32 Bit Fixed Point Implementation

% Implementation data structure
\XtoCDataStruct{Data Structure:}
\begin{lstlisting}
typedef struct {
     uint16        ID;
     int32         *In;
     int32         Out;
} LOOPBREAKER_FIP32;
\end{lstlisting}

\ifdefined \AddTestReports
\InputIfFileExists{\XcHomePath/Library/General/Doc/Test_LoopBreaker_FiP32.tex}{}{}
\fi
\XtoCImplementation{Float32}
\index{Block ID!483}
\nopagebreak[0]
% Implementation details
\begin{tabular}{l l}
\textbf{Name} & Float32 \tabularnewline
\textbf{ID} & 483 \tabularnewline
\textbf{Revision} & 0.1 \tabularnewline
\textbf{C filename} & LoopBreaker\_Float32.c \tabularnewline
\textbf{H filename} & LoopBreaker\_Float32.h \tabularnewline
\end{tabular}
\vspace{1ex}

32 Bit Floating Point Implementation

% Implementation data structure
\XtoCDataStruct{Data Structure:}
\begin{lstlisting}
typedef struct {
     uint16        ID;
     float32       *In;
     float32       Out;
} LOOPBREAKER_FLOAT32;
\end{lstlisting}

\ifdefined \AddTestReports
\InputIfFileExists{\XcHomePath/Library/General/Doc/Test_LoopBreaker_Float32.tex}{}{}
\fi
\XtoCImplementation{Float64}
\index{Block ID!484}
\nopagebreak[0]
% Implementation details
\begin{tabular}{l l}
\textbf{Name} & Float64 \tabularnewline
\textbf{ID} & 484 \tabularnewline
\textbf{Revision} & 0.1 \tabularnewline
\textbf{C filename} & LoopBreaker\_Float64.c \tabularnewline
\textbf{H filename} & LoopBreaker\_Float64.h \tabularnewline
\end{tabular}
\vspace{1ex}

64 Bit Floating Point Implementation

% Implementation data structure
\XtoCDataStruct{Data Structure:}
\begin{lstlisting}
typedef struct {
     uint16        ID;
     float64       *In;
     float64       Out;
} LOOPBREAKER_FLOAT64;
\end{lstlisting}

\ifdefined \AddTestReports
\InputIfFileExists{\XcHomePath/Library/General/Doc/Test_LoopBreaker_Float64.tex}{}{}
\fi
